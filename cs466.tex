\documentclass[]{article}
\usepackage{etex}
\usepackage[margin = 1.5in]{geometry}
\setlength{\parindent}{0in}
\usepackage{amsmath}
\usepackage{amsfonts}
\usepackage{amssymb}
\usepackage{amsthm}
\usepackage{listings}
\usepackage{color}
\usepackage{mathtools}
\usepackage{multicol}
\usepackage{pgfplots}
\usepackage{qtree}
\usepackage{xytree}
\usepackage[lined]{algorithm2e}
\usepackage{float}
\usepackage[T1]{fontenc}
\usepackage{ae,aecompl}
\usepackage[pdftex,
  pdfauthor={Michael Noukhovitch},
  pdftitle={CS 466: Design and Analysis of Algorithms},
  pdfsubject={Lecture notes from CS 466 at the Unversity of Waterloo},
  pdfproducer={LaTeX},
  pdfcreator={pdflatex}]{hyperref}

\usepackage{cleveref}
\usepackage{enumitem}

\definecolor{dkgreen}{rgb}{0,0.6,0}
\definecolor{gray}{rgb}{0.5,0.5,0.5}
\definecolor{mauve}{rgb}{0.58,0,0.82}

\lstset{
  language=C,
  aboveskip=3mm,
  belowskip=3mm,
  showstringspaces=false,
  columns=flexible,
  basicstyle={\small\ttfamily},
  numbers=none,
  numberstyle=\tiny\color{gray},
  keywordstyle=\color{blue},
  commentstyle=\color{dkgreen},
  stringstyle=\color{mauve},
  breaklines=true,
  breakatwhitespace=true,
  tabsize=4
}

\theoremstyle{definition}
\newtheorem*{defn}{Definition}
\newtheorem{ex}{Example}[section]
\newtheorem*{theorem}{Theorem}

\setlength{\marginparwidth}{1.5in}
\setlength{\algomargin}{0.75em}

\DeclarePairedDelimiter{\set}{\lbrace}{\rbrace}

\definecolor{darkish-blue}{RGB}{25,103,185}

\usepackage{hyperref}
\hypersetup{
    colorlinks,
    citecolor=darkish-blue,
    filecolor=darkish-blue,
    linkcolor=darkish-blue,
    urlcolor=darkish-blue
}
\newcommand{\lecture}[1]{\marginpar{{\footnotesize $\leftarrow$ \underline{#1}}}}

\makeatletter
\def\blfootnote{\gdef\@thefnmark{}\@footnotetext}
\makeatother

\begin{document}
	\let\ref\Cref

	\title{\bf{CS 466: Design and Analysis of Algorithms}}
	\date{Spring 2016, University of Waterloo \\ \center Notes written from Mark Petrick's lectures.}
	\author{Michael Noukhovitch}

	\maketitle
	\newpage
	\tableofcontents
	\newpage

	\section{Introduction}
	\subsection{Travelling Salesman}
	Given a graph $G = (V,E)$ with weights on edges $w: E \rightarrow \mathbb{R}^{+ve \cup \{0\}}$, find a \textbf{TSP Tour}:
	\begin{itemize}
		\item a cycle $C$ that visits evert vertex exactly once
		\item has $\min \sum_{e \in C} w(e)$
	\end{itemize}
	This is NP-Complete

	\subsection{Metrick TSP}
	Create an approximation problem with TSP by defining distances in a space:
	\begin{itemize}
		\item symmetry: $d(u,v) = d(v,u)$
		\item triangle inequality: $d(u,v) \le d(u,w) + d(w,v)$
	\end{itemize}
	\textbf{Approximation Algorithm (1977)}
	\begin{itemize}
		\item find a minimum spanning tree and walk the tour
		\item take shortcuts to avoid re-visiting the same vertices
		\item by triangle inequality, the length is $\le 2* l_{MST}$ and therefore total $l \le 2*l_{TSP}$ (2-approx)
		\item ends up with runtime of MST, O(m log n)
	\end{itemize}

	\subsection{Further Approximation}
	\begin{itemize}
		\item TSP General: no constant factor approx
		\item Metric space: 1.5-approx, proven limit is 1.0045
		\item Euclidean space: $(1+\epsilon)$-approx in polytime
	\end{itemize}

\end{document}
