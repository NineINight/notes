\documentclass[]{article}
\usepackage{etex}
\usepackage[margin = 1.5in]{geometry}
\setlength{\parindent}{0in}
\usepackage{amsmath}
\usepackage{amsfonts}
\usepackage{amssymb}
\usepackage{amsthm}
\usepackage{listings}
\usepackage{color}
\usepackage{mathtools}
\usepackage{multicol}
\usepackage{pgfplots}
\usepackage{qtree}
\usepackage{xytree}
\usepackage[lined]{algorithm2e}
\usepackage{float}
\usepackage[T1]{fontenc}
\usepackage{ae,aecompl}
\usepackage[pdftex,
  pdfauthor={Michael Noukhovitch},
  pdftitle={CS 486: Intro to AI},
  pdfsubject={Lecture notes from CS 486 at the University of Waterloo},
  pdfproducer={LaTeX},
  pdfcreator={pdflatex}]{hyperref}

\usepackage{cleveref}
\usepackage{enumitem}

\definecolor{dkgreen}{rgb}{0,0.6,0}
\definecolor{gray}{rgb}{0.5,0.5,0.5}
\definecolor{mauve}{rgb}{0.58,0,0.82}

\lstset{
  language=C,
  aboveskip=3mm,
  belowskip=3mm,
  showstringspaces=false,
  columns=flexible,
  basicstyle={\small\ttfamily},
  numbers=none,
  numberstyle=\tiny\color{gray},
  keywordstyle=\color{blue},
  commentstyle=\color{dkgreen},
  stringstyle=\color{mauve},
  breaklines=true,
  breakatwhitespace=true,
  tabsize=4
}

\theoremstyle{definition}
\newtheorem*{defn}{Definition}
\newtheorem{ex}{Example}[section]
\newtheorem*{theorem}{Theorem}

\setlength{\marginparwidth}{1.5in}
\setlength{\algomargin}{0.75em}

\DeclarePairedDelimiter{\set}{\lbrace}{\rbrace}

\definecolor{darkish-blue}{RGB}{25,103,185}

\usepackage{hyperref}
\hypersetup{
    colorlinks,
    citecolor=darkish-blue,
    filecolor=darkish-blue,
    linkcolor=darkish-blue,
    urlcolor=darkish-blue
}
\newcommand{\lecture}[1]{\marginpar{{\footnotesize $\leftarrow$ \underline{#1}}}}

\makeatletter
\def\blfootnote{\gdef\@thefnmark{}\@footnotetext}
\makeatother

\begin{document}
	\let\ref\Cref

	\title{\bf{CS 486: Intro to AI}}
	\date{Spring 2016, University of Waterloo \\ \center Notes written from Peter Van Beek's lectures.}
	\author{Michael Noukhovitch}

	\maketitle
	\newpage
	\tableofcontents
	\newpage

	\section{Introduction}
	
	\subsection{Intelligence}
	\begin{description}
		\item[Intelligence] general mental capability that includes \textit{reasoning}, \textit{planning}, \textit{thinking abstractly}.
		\item[Church-Turing Thesis] any effective computable function can be carried out on a Turing machine
		\item[Thinking] reasoning symbolically, which can according
		\item[Newell-Simon Hypothesis] A physical symbol system has the necessary and sufficient means for general intelligence
	\end{description}

	\subsection{Models of AI}
	\begin{description}
		\item[Cognitive Modelling] determine how humans think, computational theories of the mind
		\item[Turing Test] acting humanly
		\item[Laws of thought] thinking rationally
		\item[Rational agent] acting rationally based on perceptions, decision theory
	\end{description}
	The unifying theme is \textbf{intelligent agents} which percieves through sensors and outputs through actuators.

	\subsection{Design Space of AI}
	\begin{itemize}
		\item modularity
		\item repsentation scheme
		\item planning horizon
		\item uncertainty
			\begin{itemize}
				\item sensing: fully/partially observable
				\item effect: deterministic, stochastic
			\end{itemize}
		\item preference
		\item number of agents
		\item learning
		\item computational limits
	\end{itemize}

\end{document}
