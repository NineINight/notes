\documentclass[]{article}
\usepackage{etex}
\usepackage[margin = 1.5in]{geometry}
\setlength{\parindent}{0in}
\usepackage{amsmath}
\usepackage{amsfonts}
\usepackage{amssymb}
\usepackage{amsthm}
\usepackage{listings}
\usepackage{color}
\usepackage{mathtools}
\usepackage{multicol}
\usepackage{pgfplots}
\usepackage{qtree}
\usepackage{xytree}
\usepackage[lined]{algorithm2e}
\usepackage{float}
\usepackage[T1]{fontenc}
\usepackage{ae,aecompl}
\usepackage[pdftex,
  pdfauthor={Michael Noukhovitch},
  pdftitle={FRT010: Automatic Control},
  pdfsubject={Lecture notes from FRT010 at the Lund University},
  pdfproducer={LaTeX},
  pdfcreator={pdflatex}]{hyperref}

\usepackage{cleveref}
\usepackage{enumitem}

\definecolor{dkgreen}{rgb}{0,0.6,0}
\definecolor{gray}{rgb}{0.5,0.5,0.5}
\definecolor{mauve}{rgb}{0.58,0,0.82}

\lstset{
  language=C,
  aboveskip=3mm,
  belowskip=3mm,
  showstringspaces=false,
  columns=flexible,
  basicstyle={\small\ttfamily},
  numbers=none,
  numberstyle=\tiny\color{gray},
  keywordstyle=\color{blue},
  commentstyle=\color{dkgreen},
  stringstyle=\color{mauve},
  breaklines=true,
  breakatwhitespace=true,
  tabsize=4
}

\theoremstyle{definition}
\newtheorem*{defn}{Definition}
\newtheorem{ex}{Example}[section]
\newtheorem*{theorem}{Theorem}

\setlength{\marginparwidth}{1.5in}
\setlength{\algomargin}{0.75em}

\DeclarePairedDelimiter{\set}{\lbrace}{\rbrace}

\definecolor{darkish-blue}{RGB}{25,103,185}

\usepackage{hyperref}
\hypersetup{
    colorlinks,
    citecolor=darkish-blue,
    filecolor=darkish-blue,
    linkcolor=darkish-blue,
    urlcolor=darkish-blue
}
\newcommand{\lecture}[1]{\marginpar{{\footnotesize $\leftarrow$ \underline{#1}}}}

\makeatletter
\def\blfootnote{\gdef\@thefnmark{}\@footnotetext}
\makeatother

\begin{document}
	\let\ref\Cref

	\title{\bf{FRT010: Automatic Control}}
	\date{Fall 2015, Lund University\\ \center Notes written from Tore Hagglund lectures.}
	\author{Michael Noukhovitch}

	\maketitle
	\newpage
	\tableofcontents
	\newpage

	\section{Introduction}
	\subsection{Regulators}
	$e = r - y$ where
	\begin{itemize}
		\item $r$: reference signal (initial input)
		\item $y$: output signal
		\item $u$: input signal (step signal)
	\end{itemize}

	\subsubsection{On-Off Regulator}
	\[ 
		u = 
		\begin{cases}
			u_{max} & e \ge 0 \\
			u_{min} & e < 0
		\end{cases}
	\]
	you can consider $K = \infty$


	\subsubsection{P-regulator}
	\[ 
		u = 
		\begin{cases}
			u_{max} & e > e_0 \\
			u_0 + K*e & e_0 \ge e \ge -e_0 \\
			u_{min} & e < -e_0
		\end{cases}
	\]

	\subsubsection{PI-regulator}
	$P: u = u_0 + K*e $ \\
	$PI: u = K(e + \frac{1}{T_i} \int_{0}^{t} e(t)dt) $

	where $T_i$ is the regulator's integral

	\subsubsection{PID-regulator}
	$u = K(e + \frac{1}{T_i} \int edt + T_d \frac{de}{dt})$


	

	\subsection{Process Modelling}
	We can model our process with a differential eq?
	\begin{equation}
		\frac{d^ny}{dt^n} + a_1 \frac{d^{n-1}y}{dt^{n-1}} + \dots = b_0 \frac{d^nu}{dt^n} + b_1 \frac{d^nu}{dt^{n-1}} + \dots
	\end{equation}

\end{document}
