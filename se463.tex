\documentclass[]{article}
\usepackage{etex}
\usepackage[margin = 1.5in]{geometry}
\setlength{\parindent}{0in}
\usepackage{amsmath}
\usepackage{amsfonts}
\usepackage{amssymb}
\usepackage{amsthm}
\usepackage{listings}
\usepackage{color}
\usepackage{mathtools}
\usepackage{multicol}
\usepackage{pgfplots}
\usepackage{qtree}
\usepackage{xytree}
\usepackage[lined]{algorithm2e}
\usepackage{float}
\usepackage[T1]{fontenc}
\usepackage{ae,aecompl}
\usepackage[pdftex,
  pdfauthor={Michael Noukhovitch},
  pdftitle={CS 463: Software Requirements Specification and Analysis},
  pdfsubject={Lecture notes from CS 463 at the University of Waterloo},
  pdfproducer={LaTeX},
  pdfcreator={pdflatex}]{hyperref}

\usepackage{cleveref}
\usepackage{enumitem}

\definecolor{dkgreen}{rgb}{0,0.6,0}
\definecolor{gray}{rgb}{0.5,0.5,0.5}
\definecolor{mauve}{rgb}{0.58,0,0.82}

\lstset{
  language=C,
  aboveskip=3mm,
  belowskip=3mm,
  showstringspaces=false,
  columns=flexible,
  basicstyle={\small\ttfamily},
  numbers=none,
  numberstyle=\tiny\color{gray},
  keywordstyle=\color{blue},
  commentstyle=\color{dkgreen},
  stringstyle=\color{mauve},
  breaklines=true,
  breakatwhitespace=true,
  tabsize=4
}

\theoremstyle{definition}
\newtheorem*{defn}{Definition}
\newtheorem{ex}{Example}[section]
\newtheorem*{theorem}{Theorem}

\setlength{\marginparwidth}{1.5in}
\setlength{\algomargin}{0.75em}

\DeclarePairedDelimiter{\set}{\lbrace}{\rbrace}

\definecolor{darkish-blue}{RGB}{25,103,185}

\usepackage{hyperref}
\hypersetup{
    colorlinks,
    citecolor=darkish-blue,
    filecolor=darkish-blue,
    linkcolor=darkish-blue,
    urlcolor=darkish-blue
}
\newcommand{\lecture}[1]{\marginpar{{\footnotesize $\leftarrow$ \underline{#1}}}}

\makeatletter
\def\blfootnote{\gdef\@thefnmark{}\@footnotetext}
\makeatother

\begin{document}
	\let\ref\Cref

	\title{\bf{SE 463: Software Requirements Specification and Analysis}}
	\date{Spring 2016, University of Waterloo \\ \center Notes written from Joanne Atlee's lectures.}
	\author{Michael Noukhovitch}

	\maketitle
	\newpage
	\tableofcontents
	\newpage

	\section{Introduction}
	\subsection{Why}
	Software specs and requirements are necessary to prevent future repair costs (req cost post release = 200*cost during requirements phase).
	\subsection{Common Problems}
	Requirements are:
	\begin{itemize}
		\item vague 
		\item over-specified
		\item ambiguous
		\item changing
		\item incomplete
		\item infeasible
		\item contradictory
	\end{itemize}

	\section{RE Reference Model}
	\subsection{Objectives}
	Want to identify and articulate:
	\begin{description}
		\item[Requirements] conditions and/or capabilities that describe a problem (must be achieved to get solution)
			\begin{itemize}
				\item desired changes to the world
				\item expressed in terms of environment
			\end{itemize}
		\item[Specification] complete, precise, verifiable description of the proposed system
			\begin{itemize}
				\item requirement re-expressed in terms of interface
				\item no constraints on design or implementation
				\item show that specs imply requirements through \textit{assumptions}
			\end{itemize}
	\end{description}
	\subsection{Deriving Specs}
	For each requirement, $Req$, determine
	\begin{itemize}
		\item the specification, $Spec$, of how the system will monitor/control environment
		\item what domain knowledge assumptions, $Dom$ are needed to link environmental constraints to system constraints (\textit{e.g.} a plane is moving on the runway if its wheels are turning)
	\end{itemize}
	Mathematically, show that \textit{at minimum} $Spec \wedge Dom \models Req$


	\section{Stakeholders}
	\subsection{Scoping}
	\begin{description}
		\item[purpose] rationale for why the project is wanted
		\item[goal] high level measurable criteria of success
		\item[scope] business area affected by installation of system
		\item[stakeholders] people with an interest in the system
		\item[constraints] restrictions on scope or style of system
		\item[context diagram] graphical model of context
	\end{description}
	\begin{itemize}
		\item modularizes phenomena into domains
		\item system in center with arrows (actions) between it and stakeholders
	\end{itemize}

	\subsection{Stakeholders}
	\textbf{Stakeholder} someone with stake in the ultimate success of the product (but sometimes \textbf{proxies} will do)
	\begin{itemize}
		\item owner/client: person paying for software
		\item customer: buys software after development
		\item users: experts on current system or similar
		\item domain experts: familiar with problem that software might solve
		\item software engineer: techonology expert
		\item inspectors
		\item market researchers
		\item lawyers
		\item industry standards
		\item experts on \textit{adjacent} systems
		\item negative stakeholder: don't want you to succeed
	\end{itemize}

	\section{Use Cases}
	\subsection{Understanding The Work}
	\begin{itemize}
		\item understand what the existing work is
			\begin{itemize}
				\item don't think about implementation
				\item abstract inputs/outputs
			\end{itemize}
		\item study what you are permitted to change
			\begin{itemize}
				\item what you need to know to decide a change
				\item anything that can be affected by your product
			\end{itemize}
	\end{itemize}
	\subsection{Business Use Cases}
	\textbf{Use case} vertical slices of work made to reduce complexity
	\begin{itemize}
		\item represent end-to-end functionality
		\item triggered by external event
		\item captures complete response to event
		\item ideally orthogonal to one another
	\end{itemize}

	\subsection{Use Case Models}
	\begin{description}
		\item[actor generalization] actor inheritance + polymorphism
		\item[<<inlcude>>] sub use case used within multiple other use cases
		\item[<<extend>>] sub use case that extends or replaces end of other use cases
	\end{description}

\end{document}
