\documentclass[]{article}
\usepackage{etex}
\usepackage[margin = 1.5in]{geometry}
\setlength{\parindent}{0in}
\usepackage{amsmath}
\usepackage{amsfonts}
\usepackage{amssymb}
\usepackage{amsthm}
\usepackage{listings}
\usepackage{color}
\usepackage{mathtools}
\usepackage{multicol}
\usepackage{pgfplots}
\usepackage{qtree}
\usepackage{xytree}
\usepackage[lined]{algorithm2e}
\usepackage{float}
\usepackage[T1]{fontenc}
\usepackage{ae,aecompl}
\usepackage[pdftex,
  pdfauthor={Michael Noukhovitch},
  pdftitle={CS 446: Software Design and Architecture},
  pdfsubject={Lecture notes from CS 446 at the Unversity of Waterloo},
  pdfproducer={LaTeX},
  pdfcreator={pdflatex}]{hyperref}

\usepackage{cleveref}
\usepackage{enumitem}

\definecolor{dkgreen}{rgb}{0,0.6,0}
\definecolor{gray}{rgb}{0.5,0.5,0.5}
\definecolor{mauve}{rgb}{0.58,0,0.82}

\lstset{
  language=C,
  aboveskip=3mm,
  belowskip=3mm,
  showstringspaces=false,
  columns=flexible,
  basicstyle={\small\ttfamily},
  numbers=none,
  numberstyle=\tiny\color{gray},
  keywordstyle=\color{blue},
  commentstyle=\color{dkgreen},
  stringstyle=\color{mauve},
  breaklines=true,
  breakatwhitespace=true,
  tabsize=4
}

\theoremstyle{definition}
\newtheorem*{defn}{Definition}
\newtheorem{ex}{Example}[section]
\newtheorem*{theorem}{Theorem}

\setlength{\marginparwidth}{1.5in}
\setlength{\algomargin}{0.75em}

\DeclarePairedDelimiter{\set}{\lbrace}{\rbrace}

\definecolor{darkish-blue}{RGB}{25,103,185}

\usepackage{hyperref}
\hypersetup{
    colorlinks,
    citecolor=darkish-blue,
    filecolor=darkish-blue,
    linkcolor=darkish-blue,
    urlcolor=darkish-blue
}
\newcommand{\lecture}[1]{\marginpar{{\footnotesize $\leftarrow$ \underline{#1}}}}

\makeatletter
\def\blfootnote{\gdef\@thefnmark{}\@footnotetext}
\makeatother

\begin{document}
	\let\ref\Cref

	\title{\bf{CS 446: Software Design and Architecture}}
	\date{Spring 2016, University of Waterloo \\ \center Notes written from Victoria Sakhini's lectures.}
	\author{Michael Noukhovitch}

	\maketitle
	\newpage
	\tableofcontents
	\newpage

	\section{Mobile Application}
	\subsection{Overview}
	A mobile application is structured of mulitple layers: \textbf{presentation}, \textbf{business}, and \textbf{data}. 
	\subsection{Design Considerations}
	\subsubsection{Client Type}
	\begin{description}
		\item[Rich] local processing required, must work in ocassionally connected scenario
		\item[Thin] can depend on server processing and will always be fully connected
		\item[Rich Internet Application] requires a rich UI and only limited access to local resources (+ maybe portably to other platforms)
	\end{description}

	\subsubsection{Devices to Support}
	Consider
	\begin{itemize}
		\item screen size and resolution
		\item cpu power
		\item memory and storage space
		\item dev tool availability
		\item user requirements, org constraints
		\item specific hardware requirements
	\end{itemize}

	\subsubsection{Connectivity}
	If internet access is required, plan for intermittent or unavailable network connection
	\begin{itemize}
		\item caching
		\item state management
		\item batch communications
	\end{itemize}

	\subsubsection{Device Constraints}
	Think of platform constraints, mainly:
	\begin{itemize}
		\item memory
		\item battery life
			\begin{itemize}
				\item processing requirements
				\item backlighting
				\item memory I/O
				\item wireless connections
			\end{itemize}
		\item responsiveness of design
		\item security
		\item network bandwidth
	\end{itemize}

	\subsubsection{Architecture}
	\begin{itemize}
		\item layered architecture (multiple layers can be on device)
		\item reuse and maintainability
		\item smallest footprint possible
	\end{itemize}

	\subsection{Design Issues}
	\subsubsection{Authentication/Authorization}
	\begin{itemize}
		\item security and reliability
		\item think about more than single user
	\end{itemize}

	\subsubsection{Caching}
	\begin{itemize}
		\item improve performance
		\item support offline work
		\item decide on what to cache based on limited resources
	\end{itemize}
	\textbf{lazy acquisition} defer acquiring resources as long as possible

	\subsubsection{Communicaion}
	\begin{itemize}
		\item wifi, wired, bluetooth
		\item secure communication
		\item wireless is unreliable
	\end{itemize}
	\begin{description}
		\item[active object] support async processing by encapsulating service request and completion response
		\item[communicator] encapsulate internal details of communication 
		\item[entity translator] transforms message data types into business types for requests and reverses for responses
		\item[reliable sessions] end to end reliable message transfer
	\end{description}

	\subsubsection{Configuration Management}
	\begin{itemize}
		\item how to handle device resets
		\item how to allow configuration (OTA, from some host?)
	\end{itemize}

	\subsubsection{Data Access}
	\begin{itemize}
		\item low bandwidth
		\item high latency
		\item intermittent connectivity
	\end{itemize}
	\begin{description}
		\item[active record] include data access object within domain entry
		\item[data transfer object] object storing data transported between processes, reducing method calls
		\item[domain model] business objects that represent entities in a domain and relationships between them
		\item[transaction script] organize logic for each transaction in a single procedure, making calls directory to DB (or through wrapper)
	\end{description}

	\subsubsection{Device Specifics}
	\begin{itemize}
		\item screen size
		\item orientation
		\item memory, storage space
		\item network bandwidth
		\item connectiviy
		\item OS
		\item hardware constraints
	\end{itemize}

	\subsubsection{Exception Management}
	\begin{itemize}
		\item prevent sensitive exception details from being revealed to the user
		\item improve application robustness
		\item keep application in consistent state after an error
	\end{itemize}

	\subsubsection{Logging}
	\begin{itemize}
		\item log only essentials because of size constraints
		\item may need to synchronize logs with server
	\end{itemize}

	\subsubsection{Power Management}
	\begin{itemize}
		\item power is limiting design factor
		\item research communication protocols and their effect on battery life
	\end{itemize}

	\subsubsection{Synchronization}
	\begin{itemize}
		\item secure communications OTA
		\item handle connection interruptions
	\end{itemize}
	\begin{description}
		\item[sync design pattern] component installed on device tracks changes to data and tells server when connected
	\end{description}

	\subsubsection{Testing}
	Mobile debugging is costly, so make sure to invest heavily in testing beforehandas emulators may not be adequate to simulate a device in debugging.

	\subsubsection{UI}
	\begin{itemize}
		\item build mobile first
		\item design for simplicity
		\item design around blocking operations (since user can only see once screen at a time)
	\end{itemize}
	\begin{description}
		\item[application controller] object that contains all the flow logic 
		\item[MVC] separate the data, presentation, and actions into three separate classes
			\begin{itemize}
				\item model manages behaviour and data (logic)
				\item view manages information display
				\item controller manages user input
			\end{itemize}
		\item[MVP] same as MVC but presenter manages presentation logic and interaction between view and model
		\item[pagination] separate content into individual pages
	\end{description}

	\subsubsection{Validation}
	\begin{itemize}
		\item protect device and application
		\item improve usability
		\item validate client-side and server-side
	\end{itemize}

	\section{Software Architecture}
	\subsection{Definition}
	No perfect definition but AINSI/IEEE defines it as
	
	\begin{quote}
	recommended practice as the fundamental organization of a system, embodied in its components, their relationships to each other and the environment, and the principles governing its design and evolution. 
	\end{quote}

\end{document}
