\documentclass[]{article}
\usepackage{etex}
\usepackage[margin = 1.5in]{geometry}
\setlength{\parindent}{0in}
\usepackage{amsmath}
\usepackage{amsfonts}
\usepackage{amssymb}
\usepackage{amsthm}
\usepackage{listings}
\usepackage{color}
\usepackage{mathtools}
\usepackage{pgfplots}
\usepackage{float}
\usepackage[T1]{fontenc}
\usepackage{ae,aecompl}
\usepackage[pdftex,
  pdfauthor={Michael Noukhovitch},
  pdftitle={CS 348: Intro to Database Management},
  pdfsubject={Lecture notes from CS 348 at the University of Waterloo},
  pdfproducer={LaTeX},
  pdfcreator={pdflatex}]{hyperref}

\usepackage{cleveref}
\usepackage{enumitem}
\usepackage{multicol}

\definecolor{dkgreen}{rgb}{0,0.6,0}
\definecolor{gray}{rgb}{0.5,0.5,0.5}
\definecolor{mauve}{rgb}{0.58,0,0.82}

\lstset{
  language=Java,
  aboveskip=3mm,
  belowskip=3mm,
  showstringspaces=false,
  columns=flexible,
  basicstyle={\small\ttfamily},
  numbers=none,
  numberstyle=\tiny\color{gray},
  keywordstyle=\color{blue},
  commentstyle=\color{dkgreen},
  stringstyle=\color{mauve},
  breaklines=true,
  breakatwhitespace=true,
  tabsize=4
}

\theoremstyle{definition}
\newtheorem*{defn}{Definition}
\newtheorem{ex}{Example}[section]
\newtheorem*{theorem}{Theorem}

\setlength{\marginparwidth}{1.5in}

\DeclarePairedDelimiter{\set}{\lbrace}{\rbrace}

\definecolor{darkish-blue}{RGB}{25,103,185}

\usepackage{hyperref}
\hypersetup{
    colorlinks,
    citecolor=darkish-blue,
    filecolor=darkish-blue,
    linkcolor=darkish-blue,
    urlcolor=darkish-blue
}
\newcommand{\lecture}[1]{\marginpar{{\footnotesize $\leftarrow$ \underline{#1}}}}

\makeatletter
\def\blfootnote{\gdef\@thefnmark{}\@footnotetext}
\makeatother

\begin{document}
	\let\ref\Cref

	\title{\bf{CS 348: Intro to Database Management}}
	\date{Winter 2015, University of Waterloo \\ \center Notes written from Grant Weddel's lectures.}
	\author{Michael Noukhovitch}

	\maketitle
	\newpage
	\tableofcontents
	\newpage

	\section{Introduction}
		\subsection{DBMS}
			\subsubsection{Definitions}
				\textbf{Database}: a large and persistent collection of data \\
				\textbf{DBMS}: a program that manages details for storage and access to a db \\
				\textbf{Schema}: a description of the data interface to the database \\
				to abstract common functions and create a uniform interface we need:
				\begin{itemize}
					\item \textbf{data model}: all data stored uniformally
					\item \textbf{access control}: authorization to modify/view
					\item \textbf{concurrency control}: multiple applications can access at same time 
					\item \textbf{database recovery}: nothing is lost 
					\item \textbf{database maintenance}
				\end{itemize}
			\subsubsection{Three-Level Schema}
				\textbf{external schema}: what the app and user see \\
				\textbf{conceptual schema}: description of the logical structure of the data \\
				\textbf{physical schema}: description of physical aspects (storage
	algorithms \ldots{}) \\
	
				DBMS allows the data to be stored via the physical schema, reasoned via the conceptual schema, and accessed via the external schema.
	
			\subsubsection{Interfacing}
				Interfacing to DBMS, we can interact with it through: \\
				\textbf{Data Definition Language}: specifies schemas 
				\begin{itemize}
					\item may be different for each schema 
					\item the \textbf{data dictionary} (or \textbf{catalog}) stores the information
				\end{itemize}				

				\textbf{Data Manipulation Language}: specifies queries and updates \emph{(e.g SQL)}
				\begin{itemize}
					\item navigational (procedural)
					\item non-navigational (declarative)
				\end{itemize}								
				\subsubsection{DBAs}
				Database administrators are responsible for:
				\begin{itemize}
					\item managing conceptual schema
					\item assisting with app view integration
					\item monitoring and tuning DBMS performance
					\item defining internal schema
					\item loading and reformating DB
					\item security and reliability
				\end{itemize}
		\subsection{Big Ideas}
			There are three big ideas which have influenced the creation and development of databases
			\subsubsection{Quantification}
				Database queries can be described by relational algebra as quantifiers	
				
			\subsubsection{Data Independence}
				\textbf{Data Independence}: allow each schema to be independant of the others
				\begin{itemize}
					\item \textbf{physical independance}: application immune to changes
	in storage structure
					\item \textbf{logical independence}: application immune to changes in data organization
				\end{itemize}
			\subsubsection{Transaction}
				\textbf{Transaction}: an application-specified atomic and durable unit of work \\
				\textbf{ACID}: transaction properties ensured by the DBMS
				\begin{itemize}
					\item \textbf{atomic}: a transaction cannot be split up 
					\item \textbf{consistency}: each transaction preserves consistency 
					\item \textbf{isolated}: concurrent transaction don't interfere with each other 
					\item \textbf{durable}: once completed, changes are permanent
				\end{itemize}

				
	\section{Relational Model}
		\subsection{Definitions}
			\textbf{Relational model}: all information is organized in (flat) relations
			\begin{itemize}
				\item powerful and declarative query language
				\item semantic integrity constraints (using first order logic)
				\item data independence
			\end{itemize}	
		\subsection{Properties}
			\begin{itemize}
				\item based on finite set theory
					\begin{itemize}
						\item attribute ordering \textit{not strictly necessary}
						\item tuples identified by attribute values
						\item instance has set semantics \textit{no ordering, no duplicates}
					\end{itemize}
				\item all attribute values are atomic 
				\item \textbf{degree}: number of attributes in schema
				\item \textbf{cardinality}: number of tuples in instance
			\end{itemize}			 			
				
			We can algebraically define databases as a finite set of relation schemas
		\subsection{Relations vs SQL Tables}
			SQL has extensions on top of the relational model:
			\begin{enumerate}
				\item semantics of instances:
					\begin{itemize}
						\item relations are \textbf{sets} of tuples
						\item tables are \textbf{multisets} (bags) of tuples 
					\end{itemize}
				\item unknown values: SQL includes \lstinline|Null|
			\end{enumerate}
	\section{Relation Algebra}
		\subsection{Primary Operators}
			\begin{itemize}
				\item \textbf{Relation Name}: $R$
				\item \textbf{Selection}: $\sigma_{condition}(E)$ satisfies some condition
				\item \textbf{Projection}: $\pi_{attributes}(E)$ only includes these attributes
				\item \textbf{Rename}: $\rho(R(\bar{F}),E)$ (where $\bar{F}$ is a list of $oldname \mapsto newname$)
				\item \textbf{Product}: $E_1 \times E_2$
			\end{itemize}	
		\subsection{Joins}
			\begin{itemize}
				\item \textbf{Conditional Join}: $E_1 \Join_{condition} E_2$
				\item \textbf{Natural Join}: $E_1 \Join E_2$ common attributes
			\end{itemize}			
		\subsection{Set Operators}
			Schemas $R$ and $S$ must be \textbf{union compatible}: have same number (and type) of fields
			\begin{itemize}
				\item \textbf{Union}: $R \cup S$
				\item \textbf{Difference}: $R - S$
				\item \textbf{Intersection}: $R \cap S$
				\item \textbf{Division}: $R \mathbin{/} S$ (\textit{opposite of $\times$})
			\end{itemize}															
	\section{SQL}
		\subsection{SQL Standard}
			\begin{description}
				\item[Data Manipulation Language]: query and modify tables
				\item[Data Definition Language]: create tables and enforce access/security
			\end{description}				
			\begin{ex}
				Basic query block
				\begin{lstlisting}[language=SQL]
select attribute-list
from relation-list
[where condition]
				\end{lstlisting}
			\end{ex}			
		\subsection{DML}
			\subsubsection{Null}
				A necessary evil that indicates unknown or missing data
				\begin{itemize}
					\item test using \lstinline|is (not) NULL|
					\item expressions with \lstinline|NULL| e.g. \lstinline|x + NULL = NULL|
					\item \lstinline|where| treats \lstinline|NULL| like \lstinline|False|
				\end{itemize}
			\subsubsection{Subquery}
				\lstinline|where| supports predicates as part of its clause
				\begin{ex} select all employees with the highest salary
					\begin{lstlisting}[language=SQL]
select empno, lastname
from employee
where salary >= all
 	( select salary
	 from employee )				
					\end{lstlisting}
				\end{ex}
			\subsubsection{Ordering}
				No ordering can be assumed unless you use \lstinline|order by|
			\subsubsection{Grouping}
				\lstinline|group by| allows you to aggregate results
				\begin{ex} for each dept, list number of employees and combined salary
				\begin{lstlisting}[language=SQL]
select deptno, deptname, sum(salary) as totalsalary,
	count(*)as employees
from department d, employee e
where e.workdept = d.deptno 
group by deptno, deptname				
				\end{lstlisting}
				\end{ex}
				\lstinline|having| is like \lstinline|where| for groups
				\begin{ex}
					list average salary for each dept >= 4 people
					\begin{lstlisting}[language=SQL]
select deptno, deptname, avg(salary) as MeanSalary
	count(*)as employees
from department d, employee e
where e.workdept = d.deptno 
group by deptno, deptname
having count(*) >= 4
					\end{lstlisting}
				\end{ex}
	\subsection{DDL}
		\subsubsection{Table}
			\begin{description}
				\item[create]: creates a table
				\item[alter]: change the table
				\item[drop]: delete the table
			\end{description}
			\begin{ex}
				create table
				\begin{lstlisting}[language=SQL]
create table Employee (
EmpNo char(6),
FirstName varchar(12),
HireDate date
)				
				\end{lstlisting}
			\end{ex}
		\subsubsection{Data Types}
			\begin{multicols}{2}
				\begin{itemize}
					\item integer
					\item decimal(p,q)
					\item float(p)
					\item char(n)
					\item varchar(n): variable length
					\item date
					\item time
					\item timestamp: date + time
					\item year/month interval
					\item day/time interval
				\end{itemize}
			\end{multicols}
		\subsubsection{Constraints}
			\begin{itemize}
				\item not NULL
				\item primary key
				\item unique
				\item foriegn key
				\item column or tuple check
			\end{itemize}
			\begin{ex}
				add a start date that must come before hire date
				\begin{lstlisting}[language=SQL]
alter table Employee
add column StartDate date
add constraint hire_before_start
	check (HireDate <= StartDate);
				\end{lstlisting}
			\end{ex}
		\subsubsection{Triggers}
			\textbf{trigger}: procedure executed by the db in response to table change
			\begin{itemize}
				\item event
				\item condition
				\item action
			\end{itemize}
			\begin{lstlisting}[language=SQL]
create trigger log_addr
after update of addr, phone on person
referencing OLD as o NEW as n
for each row 
mode DB2SQL
when (o.status = 'VIP' or n.status = 'VIP')
	insert into VIPaddrhist(pid, oldaddr, oldphone,
		newaddr, newphone, user, modtime)
	values (o.pid, o.addr, o.phone,
		n.addr, n.phone, user, current timestamp)
			\end{lstlisting}
	\section{Views} 
		\subsection{Definition}
			\textbf{View}: a relation whose instance is determined by other relations
			\begin{itemize}
				\item \textbf{Virtual}: views not stored, used only for querying
				\item \textbf{Materialized}: query for view is executed and view is stored
			\end{itemize}
			\begin{lstlisting}[language=SQL]
create [materialized] view <name>
	as query
			\end{lstlisting}
			\begin{ex}
				Manufacturing projects view
				\begin{lstlisting}
create view ManufacturingProjects as
	( select projno, projname, firstname, lastname
	  from project, employee
	  where respemp = empno and deptno = 'D21' )
				\end{lstlisting}
			\end{ex}
		\subsection{Updating}
			Changes to a view schema propogate back to instances of relations in conceptual schema, so to avoid ambiguity a view is updateable if:
			\begin{itemize}
				\item the query references exactly one table
				\item the query only outputs simple attributes
				\item there is \textbf{no} grouping/aggregation/distinct
				\item there are no nested queries
				\item there are no set operations
			\end{itemize}
			Materialized views also have to be update with periodically to account for base table changes
		
	\section{Application Development}
		\subsection{Static Embedded SQL}		
			Embed SQL into C with \lstinline|EXEC SQL| and suffixing with \lstinline|;|, using host variables to send and recieve values from DB
			\begin{ex}
				Host variables in C
				\begin{lstlisting}[language=C]
EXEC SQL BEGIN DECLARE SECTION;
char deptno[4];
char deptname[30];
char mgrno[7];
char admrdept[4];
char location[17];
EXEC SQL END DECLARE SECTION;

/ * program assigns values to variables * /

EXEC SQL INSERT INTO
Department(deptno,deptname,mgrno,admrdept,location)
VALUES
(:deptno,:deptname,:mgrno,:admrdept,:location);
				\end{lstlisting}
			\end{ex}
			\textbf{indicator variables} are flags used to handle host variables that might recieve \lstinline|NULL|
			\begin{ex}
				Indicator variables
				\begin{lstlisting}[language=C]
int PrintEmployeePhone( char employeenum[] ) {
EXEC SQL BEGIN DECLARE SECTION;
	char empno[7];
	char phonenum[5];
	short int phoneind;
EXEC SQL END DECLARE SECTION;
	strcpy(empno,employeenum);
EXEC SQL
	SELECT phoneno INTO :phonenum :phoneind
	FROM employee WHERE empno = :empno;
if( SQLCODE < 0) { return( -1 ); } /* error */
else if(SQLCODE&=& 100){printf("no such employee\n");}
else if (phoneind<0){printf("phone unknown\n");}
else { printf("%s\n",phonenum); } 
return( 0 );
}]
				\end{lstlisting}
			\end{ex}
			\textbf{cursors}: pointer-like objects used to iterate when > 1 row is returned. Can be before the first tuple, on a tuple, after the last tuple.
			\begin{enumerate}
				\item declare the cursor
				\item open the cursor
				\item fetch tuples using the cursor
				\item close the cursor
			\end{enumerate}
		\subsection{Dynamic Embedded SQL}
			When tables, columns, predicates are not known at the time application is written
			\begin{enumerate}
				\item \lstinline|PREPARE|: prepare statement for execution
				\item \lstinline|EXECUTE|: execute the statement
				\item \textbf{placeholder}: appears instead of literals, host variables replace during execution
				\item \textbf{descriptor}: used to determine input and output numbers and types with \tt{DESCRIBE}
			\end{enumerate}
			\textbf{SQLJ}: allows embedding SQL into Java, with runtime established via JDBC connection
			\begin{ex}
				Host variables and placeholders
				\begin{lstlisting}[language=SQL]
EXEC SQL BEGIN DECLARE SECTION;
char s[100] = ``INSERT INTO employee VALUES (?, ?, ... )'';
char empno[7];
char firstname[13];
...
EXEC SQL END DECLARE SECTION;
EXEC SQL PREPARE S1 FROM :s;
strcpy(empno,``000111'');
strcpy(firstname,``Ken'');
...
EXEC SQL EXECUTE S1 USING :empno, :firstname, ... ;
				\end{lstlisting}
			\end{ex}
		\subsection{Call Level Interface}
			A vendor-neutral ISO-standard programming interface for SQL systems. As opposed to embedded SQL, does not need to be recompiled to access different DBMS and can access multiple at the same time
			\begin{itemize}
				\item queries represented as strings in the application
				\item queries prepared then executed
				\item app won't know numbers and types, descriptor areas hold metadata and actual data
				\item describing a query makes DBMS place type info into descriptor area which app can read
			\end{itemize}
		\subsection{Stored Procedure}
			Allows to execute application logic directly inside DBMS process
			\begin{itemize}
				\item minimize data transfer costs
				\item centralize application code
				\item logical independence
			\end{itemize}
			\begin{ex}
				\begin{lstlisting}[language=SQL]
CREATE FUNCTION sumSalaries(dept CHAR(3))
	RETURNS DECIMAL(9,2)
LANGUAGE SQL
RETURN
	SELECT sum(salary)
	FROM employee
	WHERE workdept = dept
				\end{lstlisting}
			\end{ex}

	\section{Data Modelling}
	\subsection{Basic ER Modelling}
	Used for database design, described in terms of:
	\begin{description}
		\item[Entity]: a distinguishable object (\textit{e.g.} student)
		\item[Attributes]: describes properties of entities (\textit{e.g.} studentName)
		\item[Relationship]: representation of two related entities (\textit{e.g.} student in course)
		\item[Role]: the function of an entity set in a relationship
	\end{description}

	\subsection{Constraints in ER Models}
	\begin{itemize}
		\item primary key
		\item relationship types (\textit{e.g.} N:1 or N:N)
		\item existence dependencies: a \textit{subordinate} entity depends on a \textit{dominant} entity
			\begin{description}
				\item[weak entity set]: entity set containing subordinate entities
				\item[strong entity set]: entity set without subordinate entities
			\end{description}
		\item cardinality constraints 
	\end{itemize}

	\subsection{Extensions}
	\begin{itemize}
		\item structured attributes:
			\begin{itemize}
				\item composite attributes: composed of a fixed number of attributes
				\item multi-valued attributes: set-valued attributes
			\end{itemize}
		\item aggregation: relationships can be viewed as higher-level attributes
		\item specialization: a more specialized``child'' entity (\textit{e.g.} graduate student)
		\item generalization: a more general ``parent'' entity (\textit{e.g.} vehicle $\rightarrow$ car, truck)
		\item disjointness: specialized entity sets are disjoint but can have entities in common
	\end{itemize}

	\subsection{Design Considerations}
	\begin{itemize}
		\item attribute or entity set: separate object $\rightarrow$ entity set
		\item entity set or relationship set
		\item degrees of relationship
		\item extended features?
	\end{itemize}

	\section{Mapping ER to Relational Tables}
	\subsection{Main Ideas}
	\begin{itemize}
		\item entity set maps to new table
		\item attribute maps to new column
		\item relationship set maps to new columns or new table
		\item specialization maps to new tables for specialized entity
		\item generalization maps to tables for specialized entities only
	\end{itemize}

	\subsection{Entity Sets}
	\begin{itemize}
		\item strong entity sets: map primary key of set to primary key of table
		\item weak entity sets: table should include attributes of identifying relationship
	\end{itemize}

	\subsection{Relationship Sets}
	\begin{itemize}
		\item if relationship is identifying for weak entity set, do nothing
		\item if we can deduce cardinality constraints, then to one entity table add columns:
			\begin{itemize}
				\item attributes of relationship set
				\item primary key of related entity sets
			\end{itemize}
		\item otherwise create a table for this relationship
	\end{itemize}

	\section{ER Schema Refinement}
	\subsection{Design Principles}
	\begin{itemize}
		\item relations should have semantic unity
		\item repetition should be avoided
		\item avoid null values
		\item avoid spurious joins
	\end{itemize}

	\subsection{Functional Dependencies}
	We want to express that some attributes uniquely determine other attributes, so if $X \rightarrow Y$
	then for any tuples $t$, $u$ where $t[X] = u[X]$, it is true that $t[Y] = u[Y]$

	\begin{description}
		\item[superkey]: set of attributes such that no two tuples have the same values for them
		\item[candidate key]: a minimal superkey
		\item[primary key]: a candidate key chosen by the DBA
	\end{description}

	\subsubsection{Closures}
	If we have a set of functional dependencies $F$, then the \textbf{functional closure} of them, $F+$, is 
	all the dependencies that can be derived from $F$. \\

	Closures can be derived using:
	\begin{itemize}
		\item reflexivity: if $X \subseteq Y$ then $Y \rightarrow X$
		\item augmentation: if $X \rightarrow Y$ then $XZ \rightarrow YZ$
		\item transitivity: if $X \rightarrow Y$ and $Y \rightarrow Z$ then $X \rightarrow Z$
	\end{itemize}

	\textbf{Attribute closure}, $X+$, is the set of all attributes functionally determined by some attribute $X$.

	\subsection{Schema Decomposition}
	We can decompose a schema $R$ into many schemas $R_1,\dots,R_n$ kind of like breaking down one big SQL table into smaller ones \\

	\begin{description}
		\item[Lossless Decomposition]: decomposing a schema so that a natural join of subschemas has no extra rows
		\item[Dependecy Preservation]: is making sure all functionally dependencies are still satisfied within our subschemas.
			Subschemas that requires fewer checks to ensure dependency preservation are better
	\end{description}

	\subsection{Normal Forms}
	\subsubsection{BCNF}
	Enforces the idea that independent relationships are stored in separate tables. A schema $R$ is in BCNF if for all $X \rightarrow Y$ either
	\begin{itemize}
		\item $X$ is a superkey of $R$ (only appears in one row) or 
		\item $Y \subseteq X$ 
	\end{itemize}

	To get BCNF, decompose original relation by removing a violating $X \rightarrow Y$ and making it into it's own relation.

	\subsubsection{Minimal Cover}
	To find the candidate key, we need to find the \textbf{minimal cover}: the minimal set of dependencies $G$ such that 
	the resulting closure is the same as our original closure $G+ = F+$. For each depenendency $X \rightarrow A$:
	\begin{itemize}
		\item every right hand $A$ is a single attribute
		\item every dependency is necessary
		\item every left hand side $X$ is minimal
	\end{itemize}

	To find the minimal cover:
	\begin{enumerate}
		\item change every $X \rightarrow YZ$ to $X \rightarrow Y, X \rightarrow Z$
		\item change $AX \rightarrow Y$ to $X \rightarrow Y$ if $X$ can determine $Y$ on its own
		\item remove $X \rightarrow Y$ if $X$ can determine $Y$ without this relation
	\end{enumerate}

	\subsubsection{3NF}
	Less strict version of BCNF, so that A schema $R$ is in BCNF if for all $X \rightarrow Y$ either
	\begin{itemize}
		\item $X$ is a superkey of $R$ (only appears in one row) or 
		\item $Y \subseteq X$ or 
		\item every attribute in $Y - X$ is in the candidate key 
	\end{itemize}

	Create 3NF in two ways:
	\begin{itemize}
		\item decomposition: decompose similar to BCNF then ``repair'' by adding all relations in minimal cover that were not satisified
		\item synthesis: turn every rule of the minimal cover into a relation and (if not already there) add a relation for candidate key
	\end{itemize}


	\section{Transactions and Concurrency}

	\subsection{Why}
	\textbf{Transaction}: application specific unit of work, ACID guaranteed
	\begin{itemize}
		\item \textbf{atomic}: a transaction cannot be split up 
		\item \textbf{consistency}: each transaction preserves consistency 
		\item \textbf{isolated}: concurrent transaction don't interfere with each other 
		\item \textbf{durable}: once completed, changes are permanent
	\end{itemize}

	They help prevent problems from failures, when the system fails during execution, and concurrency

	\subsection{Serializability}
	Concurrent transactions must appear to have been executed one at a time, everything in some order.
	\begin{description}
		\item[Equivalence]: if two transaction histories are over the same set of transactions and the ordering of each conflict pair is the same
		\item[Serializable]: a history that has an equivalent serial history (where all of $T_i$ goes before $T_j$). A history is serializable iff its
			serialization graph is acyclic
	\end{description}

	\subsection{Transactions in SQL}
	\subsubsection{Abort and Commit}
	An \textbf{active} (started but not finished) transaction can terminate in two ways:
	\begin{itemize}
		\item \textbf{abort}: transaction fails and any updates are undone (SQL \lstinline|rollback work|)
		\item \textbf{commit}: transaction succeeds and updates become durable and visible to other transactions (SQL \lstinline|commit work|)
	\end{itemize}

	\subsubsection{Isolation Levels}
	SQL allows serializability guarantee to be relaxed in four levels:
	\begin{enumerate}[start=0]
		\item Read Uncommitted: transactions can see uncommitted updates
		\item Read Committed: transactions sees only committed changes but non-repeatable reads are possible
		\item Repeatable Read: reads are repeatable but ``phantoms'' are possible
		\item Serializability
	\end{enumerate}

	\subsection{Implementing Transactions}
	\subsubsection{Concurrency Control}
	Guarantees that the execution history has desired properties (\textit{e.g serializability}) \\
	
	\textbf{Strict Two-Phase Locking}:
	\begin{itemize}
		\item before a transaction may read or write an object, it must have a lock on it
			\begin{itemize}
				\item shared lock required for read
				\item exclusive lock required for write
			\end{itemize}
		\item only one lock may be held on an object (unless all are shared locks)
		\item a transaction may not release locks until it commits
	\end{itemize}

	\textbf{Lock conflicts}, when two or more transactions request a lock, can be resolved with
	\begin{itemize}
		\item blocking: second transaction forced to wait
		\item pre-emption: first transaction is aborted and gives up the lock
	\end{itemize}

	\subsubsection{Recovery Management}
	Guarantees that committed transactions are durable (despite failures) and aborted transactions have no effect by
	\begin{itemize}
		\item rollback of individual transactions
		\item recovery from system failures
	\end{itemize}
	\textbf{System Failure}:
	\begin{itemize}
		\item database server is halted abruptly
		\item processing current SQL commands is halted abruptly
		\item connections to clients are broken
		\item contents of memory buffers are lost
		\item database files are not damaged
	\end{itemize}
	After a system failure, every transaction is either restarted or rolled back and their committed changes are not lost \\

	\textbf{Logging} is the way this is implemented, using a log in persistent storage that writes the following commands \textit{before} updating:
	\begin{itemize}
		\item UNDO information: old versions of objects used to undo changes when that transaction aborts
		\item REDO information: new versions of objects used to redo changes by a transaction that commits
		\item BEGIN/COMMIT/ABORT: records whenever a transaction does something
	\end{itemize}
	Recovering from a system failure:
	\begin{enumerate}
		\item scan the log from newest to oldest
			\begin{itemize}
				\item create list of committed transactions
				\item undo updates of active and aborted transactions
			\end{itemize}
		\item scan the log from oldest to newest
			\begin{itemize}
				\item redo updates of committed transactions
			\end{itemize}
	\end{enumerate}
	Rolling back a transaction:
	\begin{itemize}
		\item scan from the newest to the transaction's BEGIN
		\item undo the transaction's updates
	\end{itemize}

	\section{Physical Database Design and Tuning}

	\subsection{Introduction}
	\begin{description}
		\item[Physical Design]: selecting data structures to implement conceptual schema
		\item[Tuning]: periodically adjusting physical and/or conceptual schema to adapt to changing requirements or performance characteristics
		\item[Workload Description]: 
			\begin{itemize}
				\item most important queries and their frequency
				\item most important updates and their frequency
				\item performance goal for each query and update
			\end{itemize}
	\end{description}

	\subsection{Designing and Tuning the Physical Schema}
	\subsubsection{Indexing}
	A storage strategy is chosen for each relation (\textit{e.g heap}) and then we add indexes:
	\begin{itemize}
		\item substantially reduce selection time for queries with index
		\item increase insertion time
		\item increase or decrease update and deletion time
		\item increase space used to store the table
	\end{itemize}

	Create Index:
	\begin{lstlisting}[language=SQL]
create index LastnameIndex
on Employee(Lastname) [CLUSTER]
	\end{lstlisting}
	Remove Index:
	\begin{lstlisting}[language=SQL]
drop index LastnameIndex
	\end{lstlisting}

	\begin{description}
		\item[Clustering Index]: tuples with similar values are stored together in the same block. A relation can have at most one clustering index
		\item[Co-Clustering Index]: two relations with their tuples interleaved within the same file
			\begin{itemize}
				\item useful for storing heirarchical data (1:N)
				\item speeds up joins but slows down sequential scans of either
			\end{itemize}
		\item[Multi-Attribute Index]: index on more than one attribute, sorting first by the first attribute, then by second \dots
	\end{description}

	\subsubsection{Guidelines for Physical Design}
	\begin{itemize}
		\item index only when performance increase outweighs update overhead
		\item attributes mentiontioned in \lstinline|WHERE| clauses can be used for index search keys
		\item mulit-attribute search keys should be used when:
			\begin{itemize}
				\item a \lstinline|WHERE| clause contains multiple conditions 
				\item it enables index-only plans
			\end{itemize}
		\item choose indexes that benefit as many queries as possible
		\item choose clustering scheme wisely, you can only have one
			\begin{itemize}
				\item range queries benefit the most from clustering
				\item join queries benefit the most from co-clustering
				\item multi-attribute index for index-only plan doesn't benefit
			\end{itemize}
		\item there's a DB2 Index Advisor!
	\end{itemize}

	\subsection{Tuning the Conceptual Schema}
	Sometimes, even after tuning the physical schema, performance goals are not met \\

	\textbf{Denormalization}: merging schemas to intentionally increase redundancy
	\begin{itemize}
		\item decrease query overhead
		\item increase update overhead
	\end{itemize}

	\textbf{Partitioning}: splitting a large table into multiple tables to reduce I/O costs or lock contention
	\begin{itemize}
		\item \textbf{Horizontal Partitioning}: each partition has all the columns and a subset of rows
			\begin{itemize}
				\item tuples assigned based on a natural criteria
				\item often used to separate current from old data
			\end{itemize}
		\item \textbf{Vertical Partitioning}: each partition has all the rows and a subset of the columns
			\begin{itemize}
				\item used to separate frequently-used columns from each other (to improve concurrency)
			\end{itemize}
	\end{itemize}

	\subsection{Tuning Queries and Applications}
	\subsubsection{Tuning Queries}
	\begin{itemize}
		\item changes to schema impact everything, but sometimes we want to target specific queries
		\item sorting is expensive, avoid \lstinline|ORDER BY| \dots
		\item replace subqueries with joins
		\item replace correlated subqueries with uncorrelated subqueries
		\item use vendor-supplied tools to examine generated plan, update if your cost estimation sucks
	\end{itemize}

	\subsubsection{Tuning Applications}
	\begin{itemize}
		\item minimize communication costs
			\begin{itemize}
				\item return fewest columns/rows necessary
				\item update multiple rows with \lstinline|WHERE| instead of cursor
			\end{itemize}
		\item minimize lock contention and concurrency hot spots
			\begin{itemize}
				\item delay updates as long as possible
				\item delay operations on hot spots as long as possible
				\item shorten/split transactions
				\item perform insert/update/delete in batches
				\item consider lower isolation levels
			\end{itemize}
	\end{itemize}
\end{document}

	
