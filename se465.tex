\documentclass[]{article}
\usepackage{etex}
\usepackage[margin = 1.5in]{geometry}
\setlength{\parindent}{0in}
\usepackage{amsmath}
\usepackage{amsfonts}
\usepackage{amssymb}
\usepackage{amsthm}
\usepackage{listings}
\usepackage{color}
\usepackage{enumitem}
\usepackage{mathtools}
\usepackage{pgfplots}
\usepackage[lined]{algorithm2e}
\usepackage{qtree}
\usepackage{xytree}
\usepackage{float}
\usepackage[T1]{fontenc}
\usepackage{ae,aecompl}
\usepackage[pdftex,
  pdfauthor={Michael Noukhovitch},
  pdftitle={SE 465: Testing},
  pdfsubject={Lecture notes from SE 465 at the University of Waterloo},
  pdfproducer={LaTeX},
  pdfcreator={pdflatex}]{hyperref}

\usepackage{cleveref}

\definecolor{dkgreen}{rgb}{0,0.6,0}
\definecolor{gray}{rgb}{0.5,0.5,0.5}
\definecolor{mauve}{rgb}{0.58,0,0.82}

\lstset{
  language=Java,
  aboveskip=3mm,
  belowskip=3mm,
  showstringspaces=false,
  columns=flexible,
  basicstyle={\small\ttfamily},
  numbers=none,
  numberstyle=\tiny\color{gray},
  keywordstyle=\color{blue},
  commentstyle=\color{dkgreen},
  stringstyle=\color{mauve},
  breaklines=true,
  breakatwhitespace=true,
  tabsize=4
}

\lstdefinelanguage{JavaScript}{
	keywords={break, case, catch, continue, debugger, default, delete, do, else, finally, for, function, if, in, instanceof, new, return, switch, this, throw, try, typeof, var, void, while, with},
	morecomment=[l]{//},
	morecomment=[s]{/*}{*/},
	morestring=[b]',
	morestring=[b]``,
	sensitive=true
}


\theoremstyle{definition}
\newtheorem*{defn}{Definition}
\newtheorem{ex}{Example}[section]
\newtheorem*{theorem}{Theorem}

\setlength{\marginparwidth}{1.5in}
\setlength{\algomargin}{0.75em}

\DeclarePairedDelimiter{\set}{\lbrace}{\rbrace}

\definecolor{darkish-blue}{RGB}{25,103,185}

\usepackage{hyperref}
\hypersetup{
    colorlinks,
    citecolor=darkish-blue,
    filecolor=darkish-blue,
    linkcolor=darkish-blue,
    urlcolor=darkish-blue
}
\newcommand{\lecture}[1]{\marginpar{{\footnotesize $\leftarrow$ \underline{#1}}}}

\makeatletter
\def\blfootnote{\gdef\@thefnmark{}\@footnotetext}
\makeatother

\begin{document}
	\let\ref\Cref

	\title{\bf{SE 465: Testing}}
	\date{Winter 2015, University of Waterloo \\ \center Notes written from Patrick Lam's lectures.}
	\author{Michael Noukhovitch}

	\maketitle
	\newpage
	\tableofcontents
	\newpage

	\section{Introduction}
		\subsection{Types of Problems}
			\begin{itemize}
				\item \textbf{fault}: static defect in the software
				\begin{itemize}
					\item \textbf{design fault}
					\item \textbf{mechanical fault}
				\end{itemize}				 
				\item \textbf{error}: have incorrect state
				\item \textbf{failure}: external incorrect behavior
			\end{itemize}
			\begin{ex}
				Faults
				\begin{lstlisting}{language=Java}
	static public int findLast (int[] x, int y) {
		for (int i=x.length-1; i>0; --i){
	    	if (x[i] == y){
	        	return i;
	        }
	   	}
	   	return -1;
	}			
				\end{lstlisting}
				
				\textbf{fault}: should be \lstinline|i >= 0| \\
				no \textbf{fault} input: \lstinline|x = null| \\
				\textbf{fault} but not \textbf{error} input: \lstinline|x[0] != y| \\
				\textbf{error} but not \textbf{failure} input: \lstinline|y not in x|
			\end{ex}
		\subsection{RIP model}
			\textbf{RIP model}: three things necessary to observe a failure
			\begin{enumerate}
				\item \textbf{Reachability}: PC must reach that point in the program
				\item \textbf{Infection}: after fault, program state must be incorrect
				\item \textbf{Propogation}: infected state propogates to cause bad output
			\end{enumerate}
		\subsection{Dealing with faults}
			We have three ways to deal with faults:
			\begin{itemize}
				\item \textbf{avoidance}: design, use better language
				\item \textbf{detection}: testing
				\item \textbf{tolerance}: redundancy, isolation
			\end{itemize}								
	\section{Testing}
		\subsection{Testables}
			\begin{itemize}
				\item code coverage
				\item output of a function
				\item logic coverage
				\item input space coverage
			\end{itemize}
		\subsection{Types of testing}
			\textbf{static} testing: testing without running the code
			\begin{itemize}
				\item compilation
				\item semantic verification
				\item code reviews
			\end{itemize}
			\textbf{dynamic} testing: testing by running and observing the code
			\begin{itemize}
				\item \textbf{test cases}: single input, single output (wrt to some code)
				\item \textbf{black-box testing}: don't look at system implementation
				\item \textbf{white-box testing}: base tests on system's design
			\end{itemize}
		\subsection{Coverage}
			We find a reduced space and cover that space with our tests
			\begin{description}
				\item[test requirement]: a specific element (of software) that a test case must satisfy or cover
				\item[infeasable test req]: impossible coverage e.g. unreachable code
				\item[subsumption]: when one testing criterion is strictly more powerful than another criterion
			\end{description}
			
	\section{Graph Coverage}
		\begin{description}
			\item[test path]: considering our test as some path through our program from some initial node in $N_0$, along different nodes that ends up at a final node in $N_f$
			\item[subpath]: a path which is a subsequence of a path
		\end{description}
		\subsection{Behaviours}
			\begin{itemize}	
				\item \textbf{deterministic}: 	1 test path per test case
				\item \textbf{non-deterministic}: 	multiple test paths are possible
			\end{itemize}
		\subsection{Reachability}
			\begin{itemize}
				\item \textbf{syntactically}:	reachable via edges and nodes
				\item \textbf{semantically}:	there exist input that gets to a certain node
			\end{itemize}
		\subsection{Coverage Criterion}
			\begin{description}		
				\item[Node Coverage]: for every statement (node), there must be a test case that executes it
				\item[Edge Coverage]: for every branch (edge), there must be a test case that goes through it
				\item[Edge-Pair Coverage]: for every path (length <= 2), there must be a test case
			\end{description}
		\subsection{Control Flow Graph}
			The fundamental graph for source code is the \textbf{Control Flow Graph} (CFG)
			\begin{itemize}
				\item \textbf{CFG node}: zero or more statements
				\item \textbf{CFG edge}: indicates that statements follow one another
			\end{itemize}
			Group together statements that are always consecutive into a \textbf{Basic Block}, with one entry and one exit
			
	\section{Path Coverage}
		\subsection{Definitions}
			\begin{description}
				\item[simple path]: no node appears more than once in the path (first and last can be the same)
				\item[prime path]: a simple path that is not a proper subpath of any other simple path
				\item[bridge]: an edge which, when removed, results in a disconnected graph
			\end{description}
			
		\subsection{Coverage Criterion}
			\begin{description}
				\item[Complete Path Coverage]: cover paths of all lengths
				\item[Prime Path Coverage]: cover every prime path
				\item[Single Round Trip Coverage]: at least one round trip (starts = end) path for each reachable node
				\item[Complete Round Trip Coverage]: all round trip paths for each reachable node
				\item[Specified Path Coverage]: specified set of paths
				\item[Bridge Coverage]: cover all bridges
			\end{description}
	\section{Testing Concurrency}
		\subsection{Races}
			\begin{description}
				\item[Race] two concurrent accesses to the same memory and one of them is a write
			\end{description}
			
			Race freedom doesn't guarantee bug freedom, need to test code extra:
			\begin{itemize}
				\item run multiple times
				\item add noise
				\item Helgrind \ldots
				\item force scheduling
				\item static approaches
			\end{itemize}
		\subsection{Recursive Locks}
			If in one thread, there are two requests for one lock, the thread wait forever \\
			\lstinline|ReentrantLocks| know how many times they have been locked and need to be unlocked the same amount to liberate
			\begin{itemize}
				\item explicit \lstinline|lock()| and \lstinline|unlock()|
				\item \lstinline|trylock()|
			\end{itemize}
		\subsection{Bad Lock Usage}
			Lock and unlock must be paired, and comments must sufficiently describe conditions \\
			\textbf{Deadlocks} can occur if an interrupt uses the same lock as your program. 
			\begin{itemize}
				\item \lstinline|spin_lock_irqsave| disables interrupts locally and provides \lstinline|spin_lock| on symmetrical mulitiprocessors (\textbf{SMPs})
				\item \lstinline|spin_lock_irqrestore| restores interrupts to state when lock is acquired
			\end{itemize}
	\section{Assertions}
		\textbf{Assertion}: statement about the program that is true
		\begin{itemize}
			\item \textbf{precondition}: reasoning about the callee
			\item \textbf{postcondition}: reasoning about the caller
		\end{itemize}
		\subsection{Tools}
			\begin{description}
				\item Coverity
				\item iComment
				\item $\dots$
			\end{description}
		\subsection{Beliefs}
			Sometimes we do not know the truth (expected behaviour) but we can infer beliefs about how the code works. \\ 
			\newline
			\textbf{Must belief}: things that must be true in the code, any condtradiction is error
			\begin{lstlisting}[language=C]
x = *p / x	// p is not NULL
			// z != 0
			\end{lstlisting}
			\textbf{May belief}: things that correlate with each other in code
			\begin{lstlisting}[language=C]
A(); ... B();
A(); ... B();	// A and B may be paired
			\end{lstlisting}
			Check these beliefs as ``must beliefs'' and cross-check against must beliefs
			\begin{enumerate}
				\item record every sucessful MAY-belief as ``check''
				\item record every unsucessful as ``error''
				\item rank errors based on ``check'' : ``error'' ratio
			\end{enumerate}
		\subsection{Linters}
			Just because code compiles, doesn't mean it all variables are defined
			\begin{lstlisting}[language=Javascript]
function main(x) {
	if (x) {
		console.log('Yay');
	}
	else {
		console.log(num);
	}
}
main(true);
			\end{lstlisting}
			We can use \textbf{JSHint} to check that all top-level symbols resolve. On top of that we can use \textbf{pre-commit hooks} to make sure that all we checked into our repo passes the hooks. Improve further by forcing only master branch to go through all our pre-commit tests
	\section{Data Flow Criteria}
		Testing should take into account the data within nodes
		\textbf{$du$-pairs}: defintion-use pairs of nodes for variables
		\begin{lstlisting}[language=C]
int x = 5;	// definition
...
printf(x);	// use
		\end{lstlisting}
		If we consider the def line to be $n_0$ and the use $n_1$, then def($n_0$) = use($n_1$) = ${x}$
		\subsection{Defintions}
			\begin{description}
				\item[def-clear]: if our variable $v$ is not defined anywhere on our path 
				\item[use reached]: if our variable has a def-clear path from its defintion to use
				\item[$du$-path]: a simple path that is def-clear wrt $v$ from $n_i \leftarrow n_j$ where $v$ in def($n_i$) and use($n_j$)
					\begin{description}
						\item[def-path set]: fix a def and a variable, du($n_0$, $x$)
						\item[def-pair set]: fix a def, a use, and a variable, du($n_0$, $n_1$, $x$)
					\end{description}
			\end{description}
		\subsection{Graph Coverage}
			\subsubsection{Coverage Types}
				\begin{description}
					\item[All-Defs Coverage]: a test case for each def to one use
					\item[All-Uses Coverage]: a test case for each def to every use
				\end{description}
			\subsubsection{Source Code}
				\begin{description}
					\item[Def]: x is assigned, defined as paramter in method, input to program
					\item[Use]: x occurs in an expression that the program evaluates
					\item[Reachability]: if \textit{it is possible} that the address at the def refers to the same one as the use
				\end{description}
			\subsubsection{Design Elements}
				Testing beyond single methods to ``design elements'' aka \textbf{integration testing}. We use \textbf{call graphs} where design elements are nodes, calls are edges
				\begin{description}
					\item[Caller]: unit that invokes callee
					\item[Actual Parameter]: value passed to callee
					\item[Formal Parameter]: placeholder for incoming value
				\end{description}
				\hfill \\
				We also want to define new $du$-pairs between callers and callees
				\begin{description}
					\item[Last-def]: the definiton that goes through to a call
					\item[First-use]: the first use in a method, it picks up the last-def definition 
					\item[Use-clear]: a path that is clear of uses, except for at the start and end
				\end{description}
		\section{Syntax-Based Testing}
			\subsection{Input Testing}
				Use grammars to validate inputs in our program and generate test inputs.
				\subsubsection{Defintions}
					\begin{description}
						\item[Start Symbol]: the first symbol we use
						\item[Non-terminal]: symbols that are defined by production rules
						\item[Terminal]: symbols not defined by production rules
						\item[Production rule]: how a symbol is defined by other symbols
					\end{description}
					\begin{ex}
						Grammar symbols:
						\begin{lstlisting}[language=C]
actions = action*	// actions: start symbol
action = dep|deb	// action: non-terminal
dep = 'deposit' account amount		
deb = 'debit' account amount		
account = digit{3}	// = digit{3}: production rule
amount = '\$'digit+'.'digit{2}
digit = [0-9]		// 0-9: non-terminals
						\end{lstlisting}
					\end{ex}
				\subsubsection{Coverage}
					\begin{description}
						\item[Terminal Symbol Coverage]: TR contains each terminal of grammar G
						\item[Production Coverage]: TR contains each production of grammar G
						\item[Derivation Coverage]: TR contains every possible string derivable from grammar G
					\end{description}
				\subsubsection{Grammar Mutation}:
					\begin{itemize}
						\item Non-terminal Replacement
						\item Terminal Replacement
						\item Terminal and Non-terminal Deletion
						\item Terminal and Non-terminal Duplication
					\end{itemize}
			\subsection{Mutation Testing}
				Generate mutant by modifying programs and try to kill it with test cases 
				\subsubsection{Defintions}
					\begin{description}
						\item[Ground String]: a valid string in the language of the grammar
						\item[Mutation Operator]: a rule specifying synatactic variations of strings from the grammar
						\item[Mutant]: result of one application of a mutation operator to a ground string
					\end{description}
					Uninteresting Mutants:
					\begin{itemize}
						\item \textbf{Stillborn}: does not compile or immediately crashes
						\item \textbf{Trivial}: killed by almost any test case
						\item \textbf{Equivalent}: indistinguishable from original program
					\end{itemize}
				\subsubsection{Coverage}
					A test case $t$ \textbf{kills} a mutant $m$ if $t$ produces a different output on $m$ than $m_0$
					\begin{description}
						\item[Mutation Coverage]: for each mutant $m$, TR required to kill $m$
						\item[Mutation Operator Coverage]: for each mutation operator $op$, TR required to kill all mutant derived using $op$
						\item[Mutation Production Coverage]: for each mutation operator $op$ and production $p$ that it can be applied to, TR required to kill mutant from $p$
					\end{description}
				\subsubsection{Weak and Strong}
					\begin{description}
						\item[Strong Mutation]: fault must be reachable, infect state, and propogate to output
						\item[Weak Mutation]: fault which kills the mutant needs to only be reachable and infect state
						\item[Strong Mutation Coverage]: for each mutant, TR has a test that strongly kills it
						\item[Weak Mutation Coverage]: for each mutant, TR has a test that weakly kills it
					\end{description}
				\subsubsection{Mutation Operators}
					\begin{itemize}
						\item Absolute value insertion: $x > a \Rightarrow x > abs(a)$
						\item Operator replacement: $x > a \Rightarrow x < a$
						\item Scalar variable replacement: $x > a \Rightarrow x > b$
						\item Crash statement replacement: \dots
					\end{itemize}
				\subsubsection{Integration Mutation}
					\begin{itemize}
						\item Change calling method's parameters
						\item Change method being called
						\item Change inputs and outputs of called method
					\end{itemize}

						

\end{document}
