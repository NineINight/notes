\documentclass[]{article}
\usepackage{etex}
\usepackage[margin = 1.5in]{geometry}
\setlength{\parindent}{0in}
\usepackage{amsmath}
\usepackage{amsfonts}
\usepackage{amssymb}
\usepackage{amsthm}
\usepackage{listings}
\usepackage{color}
\usepackage{mathtools}
\usepackage{multicol}
\usepackage[lined]{algorithm2e}
\usepackage{float}
\usepackage[T1]{fontenc}
\usepackage{ae,aecompl}
\usepackage[pdftex,
  pdfauthor={Michael Noukhovitch},
  pdftitle={ETSF10: Internet Protocols},
  pdfsubject={Lecture notes from Jens Andersson at Lund University},
  pdfproducer={LaTeX},
  pdfcreator={pdflatex}]{hyperref}

\usepackage{cleveref}
\usepackage{enumitem}

\definecolor{dkgreen}{rgb}{0,0.6,0}
\definecolor{gray}{rgb}{0.5,0.5,0.5}
\definecolor{mauve}{rgb}{0.58,0,0.82}

\lstset{
  language=C,
  aboveskip=3mm,
  belowskip=3mm,
  showstringspaces=false,
  columns=flexible,
  basicstyle={\small\ttfamily},
  numbers=none,
  numberstyle=\tiny\color{gray},
  keywordstyle=\color{blue},
  commentstyle=\color{dkgreen},
  stringstyle=\color{mauve},
  breaklines=true,
  breakatwhitespace=true,
  tabsize=4
}

\theoremstyle{definition}
\newtheorem*{defn}{Definition}
\newtheorem{ex}{Example}[section]
\newtheorem*{theorem}{Theorem}

\setlength{\marginparwidth}{1.5in}
\setlength{\algomargin}{0.75em}

\DeclarePairedDelimiter{\set}{\lbrace}{\rbrace}

\definecolor{darkish-blue}{RGB}{25,103,185}

\usepackage{hyperref}
\hypersetup{
    colorlinks,
    citecolor=darkish-blue,
    filecolor=darkish-blue,
    linkcolor=darkish-blue,
    urlcolor=darkish-blue
}
\newcommand{\lecture}[1]{\marginpar{{\footnotesize $\leftarrow$ \underline{#1}}}}

\makeatletter
\def\blfootnote{\gdef\@thefnmark{}\@footnotetext}
\makeatother

\begin{document}
	\let\ref\Cref

	\title{\bf{ETSF10: Internet Protocols}}
	\date{Fall 2015, Lund University \\ \center Notes written from Jens Andersson's lectures}
	\author{Michael Noukhovitch}

	\maketitle
	\newpage
	\tableofcontents
	\newpage

	\section{Internet Routing}
	\subsection{Routing}
	\textbf{Routing} select route across network between nodes, requiring:
	\begin{itemize}
		\item correctness
		\item simplicity
		\item robustness
		\item \dots
	\end{itemize}

	\subsubsection{Flooding}
	\textbf{Flooding}: packets are sent by node to every neighbour and eventually at least one copy arrives at the destination
	\begin{itemize}
		\item no network information required
		\item uniquely number packets, so we can discard duplicates
		\item limit infinite transmission with time-to-live
	\end{itemize}

	\subsubsection{Packet-Switching}
	\textbf{Packet-switching}: choose optimal path according to a cost metric, make it decentralized

	\subsection{Router Architechture}
	\textbf{Router}: internetwork device that passes data between networks, by checking network layer addresses
	\begin{itemize}
		\item routing
		\item forwarding
	\end{itemize}

	\textbf{Input port}: getting input from line termination to the switch fabric
	\begin{itemize}
		\item physical layer: bit-level reception
		\item data link layer
		\item switching: look up output port using routing table in input port memory 	
		\item queuing: for fabric slower than input
			\begin{itemize}
				\item delay and loss from overflow
				\item Head-of-the-line blocking
			\end{itemize}
	\end{itemize}

	\textbf{Output port}: outputting packets to physical layer
	\begin{itemize}
		\item priority scheduling: schedule most important packets to leave first 
	\end{itemize}

	\subsection{Best Path}
	\textbf{Performance criteria}: used for selection of route
	\begin{itemize}
		\item minimum hop
		\item least cost (more flexible)
	\end{itemize}

	Decision time:
	\begin{itemize}
		\item packet or virtual circuit basis
		\item fixed or dynamically changing
	\end{itemize}

	Decision place:
	\begin{itemize}
		\item distributed: made by each node
		\item centralized: made by a designated node
		\item source: made by source station
	\end{itemize}

	Network information:
	\begin{itemize}
		\item distributed routing: using only local knowledge, information from adjacent nodes, maybe information from all nodes on a potential route
		\item central routing: information from all nodes
	\end{itemize}

	Update timing:
	\begin{itemize}
		\item fixed: never updated
		\item adaptive: regular updates
	\end{itemize}

	\subsection{Routing Strategies}
	\subsubsection{Basic Types}
	\textbf{Fixed routing}: single permanent route for each source-dest pair (uses least-cost)
	\begin{itemize}
		\item fixed until a change in network topology
		\item simple but inflexible
	\end{itemize}
	\textbf{Adaptive routing}: routing decisions change as congestion and failure happen, classified based on information source
	\begin{itemize}
		\item local: route to outgoing link with shortest queue
		\item adjacent: use delay and outage info
		\item all nodes: like adjacent
	\end{itemize}

	\subsubsection{Network Types}
	\begin{description}
		\item[Autonomous System] a set of connected routers managed by single org under same protocol 
		\item[IRP] Interior Router Protocol, routing inside AS
		\item[ERP] Exterior Router Protocol, routing between AS
	\end{description}

	\subsubsection{Distance-Vector}
	Requires that each node exchange info with neighbouring nodes, maintaing a forwarding table with distance (cost), destination, and next-hop.

	\textbf{RIP} (Routing Information Protocol) uses this with:
	\begin{itemize}
		\item max 15 hops
		\item updates exchanged every 30s
		\item implemented in UDP
	\end{itemize}

	Updates are given in the form of the whole forwarding table, to be incremented and upsert current entries. Problems can arise from ``count to infinity'', solved by ``poison reverse''

	\subsubsection{Link-State Routing}
	Determine link costs of neighbour nodes and advertise this to \textit{ALL} nodes (so everyone knows topology)







\end{document}
